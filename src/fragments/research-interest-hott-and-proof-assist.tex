\section*{Research Interests}

I am deeply intrigued by the foundations of math.
I'm reading books in set theory, proof theory, and logic.
Type theory,
an alternative yet impactful foundation system of mathematics, has been my recent favorite area of study.
As of this application, I have been actively reading and doing exercise problems in Homotopy Type Theory.
I find many of its concepts intriguing because of my familiarity with the Rust programming language,
a functional programming language with applied ML type theory.

In general, I find type theory immensely beautiful in two ways,
both philosophically and pragmatically.

First there is the intuitionist philosophy behind.
It always touches my soul when I think of the debates that
happened in the early 1900s,
where foundations of mathematics were being built
and choices in the logic of mathematics were being made,
when set theory was chosen to prevail.
Yet things have changed a lot since then.
As Gödel shows in the incompleteness theorems,
theories are not perfect in their proving power.
Alternative theories can provide the strength of different proving powers.
On the other hand, the nature of intuitionist logic used in type theory always feels appealing to me.
The omission of the law of excluded middle in type theory, for example,
seems to fit well with the interpretations brought by contemporary physics,
which Hilbert and mathematicians back then didn't have access to.
Just like how software gets better with refactoring and iterations,
with the benefit of hindsight,
reconstructions of the foundations of mathematics from a modern perspective
will bound to bring benefits.

Secondly, the constructive nature of type theory serves as a bridge
between computers and proofs.
Type theory is the underlying theory in proof assistants, programs that understand proofs.
With proof assistants,
math theorems can be completely computerized and checked meticulously by the machine.
It's just exciting to imagine every math theorem so neatly formalized and their correctness checked.
While I am learning Lean,
a proof assistant with increasing popularity,
I'm amazed at the community of mathematicians implementing advanced math concepts in code.
With recently advances in human-machine interaction brought by type theory,
proof assistants is really becoming a feasible research tool as well as an education tool in mathematics.
And I believe it's bound to revolutionize how math is researched in the future.
Just as what Dr. Andrej Bauer said at the 8th European Congress of Mathematics in 2021:

\begin{displayquote}
    But will formalized mathematics go mainstream?
    I think this is the completely wrong question to answer, \ldots of course it will \ldots
    the question is, WHEN \ldots \footnote{video: \url{https://youtu.be/zp6WleEjHUg?t=2279} timestamp relevant}
\end{displayquote}

Another person who inspired my a lot is Dr. Kevin Buzzard at Imperial College London,
who is applying proof assistants to both undergraduate math education and advanced research topics like algebraic geometry.
It was him and his Xena project that made a tutorial for Lean on the internet and lead me to the world of proof assistants.
Dr. Buzzard and I even had a small discussion on Zulip where he answered my questions with great kindness and patience.

If there's a chance,
I wish to become an evangelist like him and integrate proof assistants as an inherent part of my future research.
I wouldn't doubt in the future, coding with proof assistants will become as important as \LaTeX\ for mathematicians.
And I want to contribute to this future.

``I am excited by the vision of a future where computers can do research by themselves''
were the words I wrote back then on my application to undergraduate schools.
As crazy as it sounded like, it's partially becoming the reality.
It's more so as researchers are exploring the topic of combining machine learning and proof assistants.
One of the recent news is from OpenAI which built a ``neural theorem prover'' for lean that was able to solve
a variety of challenging high-school olympiad problems\footnote{the solver from OpenAI: \url{https://openai.com/blog/formal-math/}}.
