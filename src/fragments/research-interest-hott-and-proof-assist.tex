\section*{Research Interests}

Having a career and in academic research is always my biggest dream.
Yet not being able to decide on a specific area,
I didn't hurry to apply to a school,
but instead spent lots of time surveying different areas of mathematics.
Not too late, I found myself deeply intrigued by the foundations of math.
I started reading books in set theory, proof theory, and logic.
This is when I inevitably came across homotopy type theory (HoTT),
an alternative yet impactful foundation system of mathematics
and an innovation in type theory.
I find many of its concepts familiar thanks to my familiarity
with the Rust programming language,
a functional programming language with applied ML type theory.

I find type theory immensely beautiful in two ways,
both philosophically and pragmatically.

First there is the intuitionist philosophy behind.
It always touches my soul when I think of the debates that
happened in the early 1900s,
where foundations of mathematics were being built
and choices in the logic of mathematics were being made,
when set theory was chosen to prevail.
Yet things have changed a lot since then.
As Gödel shows in the incompleteness theorems,
theories are not perfect in their proving power.
Alternative theories can provide the strength of different proving powers.
On the other hand, the nature of intuitionist logic used in type theory always feels appealing to me.
The omission of the law of excluded middle in type theory, for example,
seems to fit well with the interpretations brought by contemporary physics,
which Hilbert and mathematicians back then didn't have access to.
Just like how software gets better with refactoring and iterations,
with the benefit of hindsight,
reconstructions of the foundations of mathematics from a modern perspective
will bound to bring benefits.

Secondly, the constructive nature of type theory serves as a bridge
between computers and proofs.
Type theory are the underlying theory in proof assistants, programs that understand proofs.
With proof assistants,
math theorems can be completely computerized and checked meticulously by the machine.
It's just exciting to imagine every math theorem so neatly formalized and their correctness checked.
While I am learning Lean,
a proof assistant with increasing popularity,
I'm amazed at the community of mathematicians implementing advanced math concepts in code.
With recently advances in human-machine interaction brought by type theory,
proof assistants is really becoming a feasible research tool as well as an education tool in mathematics.
And I believe it's bound to revolutionize how math is researched in the future.
Just as what the Dr. Andrej Bauer said in 2021:

\begin{displayquote}
    But will formalized mathematics go mainstream?
    I think this is the completely wrong question to answer, \ldots of course it will \ldots
    the question is, WHEN \ldots \footnote{video: \url{https://youtu.be/zp6WleEjHUg?t=2279} timestamp relevant}
\end{displayquote}

Another person who inspired my a lot is Dr. Kevin Buzzard at Imperial College London,
who is applying proof assistants to both undergraduate math education and in advanced research topics like algebraic geometry.
If there's a chance,
I wish to become an evangelist like him and integrate proof assistants as an inherent part of my future research.
I wouldn't doubt in the future, coding will become as important as \LaTeX\ for mathematicians.
And I want to contribute to this future.

``I am excited by the vision of a future where computers can do research by themselves''
were the words I wrote back then on my application to undergraduate schools.
As crazy as it sounded like, it's partially becoming the reality,
not mentioning researchers exploring further on this idea by applying AI to proof assistants.
